% Took this from CogSci but removed the header, sorry.
%
% Author: Matthew Turner
% Date: 2017-11-23

\documentclass[11pt,letterpaper]{article}

% \usepackage{cogsci}
\usepackage{fullpage}
\usepackage{booktabs}
\usepackage{pslatex}
\usepackage{apacite}
\usepackage{amsmath}
\usepackage{subcaption}
\usepackage{pgfplots}
\usepackage{wrapfig}
\usepackage{url}
\usepackage{hyperref}
\usepackage{bigfoot}
\usepackage[export]{adjustbox}
\setlength\intextsep{0pt}

\usepackage{graphicx}

\usepackage{gb4e}  % linguistic examples
\noautomath

\usepackage{amsthm}
\newtheorem*{remark}{Remark}

\title{Notes on Norbert Wiener's \emph{Cybernetics}}

\author{{\large \bf Matthew A.~Turner (mturner8@ucmerced.edu)}}

\begin{document}
\maketitle

\section{Ergodic theory, the Fourier integral as an invariant under translations}
\label{sec:label}

A \emph{character} of a transformation group, where transformations $T$ transform
coordinates $x$, are functions $f(x)$ such that 

\begin{equation*}
  f(Tx) = \alpha(T) f(x)
\end{equation*}
\noindent
where \(|\alpha(T)| = |f(x)| = 1\) for all $T$ and $x$. On p. 51, Wiener remarks,
``clearly $f(x)g(x)$'' is a character if $f(x)$ and $g(x)$ are characters.
Here is a proof of this assertion.

\begin{remark}
    If $f(x)$ and $g(x)$ are characters, then $h(x) = f(x)g(x)$ is 
    also a character of the group.
\end{remark}
\begin{proof}
    \begin{align*}
        h(Tx) &= f(Tx)g(Tx) \\
              &= \alpha_f(T)f(x) \cdot \alpha_g(T)g(x)
    \end{align*}

    Let \(\alpha_h(T) = \alpha_f(T)\alpha_g(T)\). Then 
    \begin{align*}
        |\alpha_h(T)| &= |\alpha_f(T)\alpha_g(T)| \\
                      &= |\alpha_f(T)| \cdot |\alpha_g(T)| \\
                      &= 1
    \end{align*}

    Similarly, since the absolute value of $f(x)$ and $g(x)$ are both 1, 
    \(|h(x)| = 1\). We then have $h(Tx) = \alpha_h(T)h(x)$, where the
    $|\alpha_h(T)| = 1$ and $|h(x)| = 1$. Therefore $h(x)$ is also a character of
    the transformation group.
\end{proof}

\nocite{Wiener1961}

\bibliographystyle{apacite}

\setlength{\bibleftmargin}{.125in}
\setlength{\bibindent}{-\bibleftmargin}
\bibliography{/Users/mt/workspace/papers/library.bib}

\end{document}
